% LaTeX resume using res.cls
\documentclass[margin]{res}
%\usepackage{helvetica} % uses helvetica postscript font (download helvetica.sty)
%\usepackage{newcent}   % uses new century schoolbook postscript font 
\setlength{\textwidth}{6.2in} % set width of text portion
\newcommand{\code}[1]{\texttt{#1}}
\usepackage{hyperref}
\hypersetup{
    colorlinks=true,       % false: boxed links; true: colored links
    linkcolor=red,          % color of internal links (change box color with linkbordercolor)
    citecolor=green,        % color of links to bibliography
    filecolor=magenta,      % color of file links
    urlcolor=blue           % color of external links
}

\begin{document}

% Center the name over the entire width of resume:
 \moveleft.5\hoffset\centerline{\textsc{\Large Brett M. Morris}}
 \moveleft.5\hoffset\centerline{Curriculum Vitae}

% Draw a horizontal line the whole width of resume:
%\moveleft\hoffset\vbox{\hrule width\resumewidth height 1pt}\smallskip
% \moveleft.5\hoffset\centerline{Email: \href{mailto:bmmorris@uw.edu}{bmmorris@uw.edu}}
% \moveleft.5\hoffset\centerline{Department Website: \url{http://staff.washington.edu/bmmorris}}
\moveleft\hoffset\vbox{\hrule width\resumewidth height 1pt}

\hspace{-1in}
\begin{minipage}[t]{0.95\resumewidth}
Email: \href{mailto:bmmorris@uw.edu}{bmmorris@uw.edu} \hfill GitHub: \href{https://github.com/bmorris3}{bmorris3}\\
Web: \url{http://staff.washington.edu/bmmorris} \hfill 
\end{minipage}
%% Select character to use as the bullet marker. 
%% Comment out to get normal dark filled circle bullets
\def\labelitemi{--}	

\begin{resume}

\section{Education} 
University of Washington, Seattle, WA \hfill 2014 -- present\\
Graduate student in Astronomy and Astrobiology (dual-title PhD program)

University of Washington, Seattle, WA \hfill 2013 -- 2014\\
M.S.\ in Astronomy 

University of Maryland, College Park, MD  \hfill 2009 -- 2012\\
B.S.\ with High Honors in Astronomy \\
B.S.\ in Physics (double degree)\\

\section{Selected \\Graduate \\Research} 
{\sl Principle Investigator}: Two nights at Keck Observatory (MOSFIRE)   \hfill          2014\\
On transit transmission spectroscopy of giant exoplanet atmospheres in the near-infrared
\begin{itemize}  \itemsep -2pt %reduce space between items
\item Developed observing, data reduction, and analysis techniques for transmission spectroscopy of giant exoplanet atmospheres
\item Achieved spectrophotometric precision of $<$2$\times$ the photon noise floor  (Morris et al.~2016, in prep.)
\end{itemize}  

{\sl Principle Investigator}: 70+ nights at Apache Point Observatory (Agile, FlareCam) \hfill          2014 -- present\\
Pulsation photometry and transiting planet search targeting metal-polluted white dwarfs (``SPAMS'')
\begin{itemize}  \itemsep -1pt %reduce space between items
\item Monitoring newly classified, metal-polluted, ZZ Ceti white dwarfs for pulsations and transiting planets/planetary debris with the ARC 3.5 m and ARCSAT 0.5 m telescopes
\item Mentoring undergraduate \href{http://www.astro.washington.edu/users/premap/}{Pre-MAP} students to reduce, analyze the data for credit
\end{itemize}  

Stellar astrophysics via starspot occultations by HAT-P-11 b   \hfill          2015--present\\
Using Kepler observations of richly spotted exoplanet host HAT-P-11 to study stellar astrophysics
 \begin{itemize}  \itemsep -2pt %reduce space between items
 \item Modeling starspot positions along transit chord in 200 transits using an original photometric inversion model and a forward modelling approach developed by Prof.\ Leslie Hebb (HWSC)
 \item Measured distribution of starspot positions,  characterized an active stellar latitude
 \end{itemize}  
             
{\sl Google Summer of Code}: Developer, maintainer of \texttt{astroplan}\footnote{\url{https://github.com/astropy/astroplan}} \hfill          2015 -- present\\
Co-wrote and presently maintain an \texttt{astropy}-affiliated package for observation planning 
 \begin{itemize}  \itemsep -1pt %reduce space between items
 \item Provides the first observation planning toolkit in Python built on the open source \texttt{astropy} ecosystem of Python packages for astronomers
 \item Funded by the Python Software Foundation, \texttt{astroplan} was presented
 at the .Astronomy conference in Sydney, Australia in November 2015\\
 \end{itemize}  

\section{Employment} 

{\sl NASA Goddard Space Flight Center Research Assistantship} \hfill           Jan 2013 -- Aug 2013\\
Post-baccalaureate research assistantship with advisor Dr.\ Avi Mandell at the Goddard Center for Astrobiology.
\begin{itemize}%  \itemsep -2pt %reduce space between items
\item Prepared a Python data reduction pipeline for near-infrared differential spectrophotometric observations with Keck/MOSFIRE and Keck/NIRSPEC of transiting exoplanet atmospheres.\\
\end{itemize}  

\section{Under-\\graduate\\Research} 
{\sl NASA Goddard Space Flight Center Research Associateship in Astrobiology} \hfill            Jun 2012 -- Aug 2012 \\
Undergraduate research associateship with mentor Dr.\ Avi Mandell at the Goddard Center for Astrobiology.
\begin{itemize}%  \itemsep -2pt %reduce space between items
\item Wrote original Python algorithms to compute differential photometry of transiting exoplanet HAT-P-7b to detect the secondary eclipse using near-infrared observations from the Hale Telescope at the Palomar Observatory. 

\item Generated composite light curves with Kepler photometry to measure the orbital parameters and atmospheric properties of HAT-P-7b, with original analysis code in Python and IDL. 

\item Proposed the first evidence for planet-induced stellar gravity darkening in the HAT-P-7 system.
\end{itemize}  


{\sl Undergraduate Research with Professor Drake Deming} \hfill            Aug 2011 -- Dec 2012 \\
Independent research for credit (ASTR498), three semesters. 
\begin{itemize}
\item Traveled to Arizona to act as co-investigator on an observing campaign on the 2.1m telescope at Kitt Peak National Observatory for transiting exoplanet observations in the near-infrared. 

\item Developed observing techniques and wrote Python algorithms to compute differential photometry of transiting extrasolar planets using observations obtained at the UMD Observatory, Kitt Peak National Observatory and Palomar Observatory. 

\item Made the first exoplanet transit light curves produced at the UMD Observatory.

\item Released and maintain an open source differential photometry code for undergraduate and serious amateur astronomers called ``OSCAAR\footnote{\url{http://oscaar.github.io}}''.

\item Submitted transit light curves to the Czech Astronomical Society Exoplanet Transit Database recorded at the University of Maryland Observatory.
\end{itemize} 

{\sl Department of Astronomy Senior Honors Thesis} \hfill Aug 2010 -- Dec 2012 \\
``Numerical Modeling of Rotational Fission of Contact Binary Asteroids'' with Professor Derek Richardson, five semesters. 
\begin{itemize}   
\item Modeled binary asteroid systems in Python and interfaced with an N-body integrator to probe the relative configuration equilibria of continuously torqued 
contact binary asteroid systems. Used resources at the Department of Astronomy's Center for Theory and Computation 
and UMD's High Performance Computing Cluster, computed for more than 4 years of CPU time. 
\item Cited with ``High Honors'' by the Department of Astronomy for this work.\\
\end{itemize}

%{\sl Undergraduate Research with Prof.\ Christopher Reynolds and Dr.\ Sean O'Neill} \hfill        Spring 2010 \\
%Visualization of magneto-rotational instability in model black hole accretion disks, one semester.
%\begin{itemize}  
%\item Produce visualizations with VisIt to emphasize relevant astrophysical processes in model black hole accretion disks in Python. \\
%\end{itemize} 

\section{Publications}
\textit{First author:}
\begin{itemize}   
\item \textbf{Morris, B.M.}, Mandell, A.M., Deming, D. ``\href{http://adsabs.harvard.edu/abs/2013ApJ...764L..22M}{Kepler's Optical Secondary Eclipse of HAT-P-7b and Probable Detection of Planet-Induced Stellar Gravity Darkening}.'' The Astrophysical Journal Letters, Volume 764, Issue 2, article id.\ L22, 5 pp.\ (2013).
\end{itemize}
\textit{n\textsuperscript{th} author:}
\begin{itemize}   
\item Hallakoun, N.; Maoz, D.; Kilic, M.; Mazeh, T.; Agol, E.; Bell, K. J.; Bloemen, S.; Brown, W. R.; Debes, J.; Faigler, S.; Gianninas, A.; Kull, I.; Kupfer, T.; Loeb, A.; \textbf{Morris, B. M.}; Mullally, F. ``\href{http://adsabs.harvard.edu/abs/2015arXiv150706311H}{SDSS J1152+0248: An eclipsing double white dwarf from the Kepler K2 campaign}.'' Submitted, arXiv:1507.06311.\\
\end{itemize}

\section{Honors\\And\\Awards} 
\begin{itemize}  
\item Pacific Science Center Science Communication Fellow (2016)

\item Chambliss Astronomy Achievement Graduate Student Award Honorable Mention. 225\textsuperscript{th} AAS, Seattle, WA (2015), and 222\textsuperscript{nd} AAS, Indianapolis, IN (2013).
 
\item Astrobiology Fellow, University of Washington, 2013-2014.

\item ``Audience Choice Award'' at the \textit{Astrobiology Speaks!} public outreach speaking competition. AbGradCon 2013, McGill University, Montreal, Canada. 

\item Invited to present thesis for Department of Astronomy Honors, awarded High Honors citation (2012).\\
\end{itemize}
                 
\section{Observing\\Experience}
\begin{itemize}   
\item {\bf Principle investigator} on Keck Observatory/MOSFIRE proposal: ``Probing Giant Planet Formation with MOSFIRE Exoplanet Transmission Spectroscopy'', awarded 2 nights (2014)

\item {\bf Co-investigator} on Very Large Telescope/KMOS proposal: ``Exoplanet transits with KMOS: Is GJ 1214b a water-world Super Earth or a cloudy Mini-Neptune?'', awarded 2 nights (PI: D. Angerhausen, 2014)

\item {\bf Co-investigator} on Keck Observatory/MOSFIRE proposal: ``Comprehensive Characterization of CoRoT-2b and XO-1b with Keck Observatory/MOSFIRE'', awarded 2 nights (PI: A. Mandell, 2013)

\item {\bf Co-investigator} on Kitt Peak National Observatory 2.1m/FLAMINGOS proposal: ``A Near-infrared Exoplanet Transit and Eclipse Survey'', awarded 6 nights (PI: D. Deming, 2012)

\item Undergraduate research at the University of Maryland Observatory, 152 mm (2010-2013): $>100$ hours collecting photometry of transiting exoplanets and asteroids. \\
\end{itemize}



\section{Professional\\ Presentations} 
\begin{itemize}  %\itemsep -5pt %reduce space between items

\item Poster: ``\href{http://adsabs.harvard.edu/abs/2015AAS...22525710M}{Exoplanet Transmission Spectroscopy in the Near-Infrared with Keck/MOSFIRE}.'' 225\textsuperscript{th} American Astronomical Society Meeting. Seattle, WA. January 6, 2015.

\item Poster: ``\href{http://nexsci.caltech.edu/conferences/KeplerII/posters/morris.pdf}{Kepler's Optical Secondary Eclipse of HAT-P-7b and Probable Detection of Planet-Induced Stellar Gravity Darkening}.'' Second Kepler Science Conference, NASA Ames Research Center, Mountain View, CA. November 6, 2013. 

\item Talk: ``\href{http://youtu.be/ZMfbkCzzQUE}{Kepler's Optical Secondary Eclipse of HAT-P-7b and Probable Detection of Planet-Induced Stellar Gravity Darkening}''. AbGradCon 2013, McGill University, Montreal, Canada. June 11, 2013.

\item Poster: ``\href{http://adsabs.harvard.edu/abs/2013AAS...22221717M}{Differential Photometry with OSCAAR: Open Source Differential Photometry Code for Amateur Astronomical Research}''. 222\textsuperscript{nd} American Astronomical Society Meeting. Indianapolis, IN. June 4, 2013.

\item Talk: ``\href{https://astrobiology.nasa.gov/seminars/featured-seminar-channels/gsfc-summer-internship/2012/08/09/gsfc-summer-research-associate-presentations/}{Secondary Eclipse and Transiting Timing of Extra-Solar Planet HAT-P-7b}.'' 2012 NASA Goddard Space Flight Center Summer Research Associate Symposium. Goddard Center for Astrobiology, NASA Goddard Space Flight Center, Greenbelt, MD, August 9, 2012.

\item Undergraduate Thesis defense:  ``Numerical Simulations of Rotational Fission of Contact Binary Asteroids.'' Department of Astronomy, University of Maryland, College Park, MD. May 4, 2012.\\
\end{itemize}



\section{Press} 
\begin{itemize}
\item {\it Press release:} ``\href{http://www.nasa.gov/content/nasa-funded-program-helps-amateur-astronomers-detect-alien-worlds/}{NASA-funded Program Helps Amateur Astronomers Detect Alien Worlds}''. NASA Goddard Space Flight Center, Greenbelt, Md. September 4, 2013.

\item Feature article in the UMD ``\href{http://www.scholars.umd.edu/news/newsletter/363-february-2012-issue-1}{Scholars Newsletter}'' for research achievements (Feb 2012).\\
\end{itemize}              


\section{Teaching \\Experience}
\begin{itemize}   
\item ASTR150 The Planets: Teaching assistant for three quarters (Fall 2013, Spring 2014, Spring 2015).

\item ASTR101 Intro Astronomy: Teaching assistant for one quarter (Winter 2014). \\
\end{itemize}

\section{Mentorship}
\begin{itemize}
\item 2014-present: Formed the Search for Planets Around post-Main Sequence stars (SPAMS) research group with five undergraduates in the University of Washington's Pre-Major in Astronomy Program (\href{http://www.astro.washington.edu/users/premap/}{Pre-MAP}), which searches for transiting planetary material orbiting white dwarfs
\item 2015-2016: Academic mentor (paid position) for Pre-MAP Cohort 11\\
\end{itemize}

\section{Consulting}
{\sl Center for Inquiry Science at the Institute for Systems Biology, Seattle, WA}\hfill2014-2015\\
Bringing professional STEM experience to middle school classrooms in the Puget Sound area
\begin{itemize}   
\item Worked with school science teachers in Renton School District to adapt their curriculum to comply with new state standards as part of the Partnership in Science and Engineering Practices project. 
\item Collaborated with science teachers at Meeker Middle School (Tacoma, WA) to update a Sun-Moon-Earth system lab as part of the Observing for Evidence of Learning professional development model.\\
\end{itemize}

\section{Public \\ Outreach}             
\begin{itemize}
\item Co-founded and co-host of the Seattle branch of Astronomy on Tap (2015-present)

\item Developed open source differential photometry routine with educational documentation for amateur transiting exoplanet observations (OSCAAR).

\item President of the ``AstroTerps'' (2011-2012), the UMD Astronomical Society.

\item Volunteer for Visitor Services at the Smithsonian Air and Space Museum in Washington, D.C., May -- Aug 2011\\
\end{itemize}


\section{Public Talks}
\begin{itemize}
\item ``Pluto-Palooza -- a First Look at Images from New Horizons.'' Astronomy on Tap, Bad Jimmy's Brewery, Seattle, WA. July 15, 2015.

\item ``Dear Grandpa: Here's What NASA Actually Does.'' Astronomy On Tap, Bad Jimmy's Brewery, Seattle, WA. March 11, 2015.

\item ``Exoplanets and Astrobiology.'' University of Washington Planetarium Open House, Seattle, WA. February 6, 2015.

\item ``Exoplanets: How to find them and their inhabitants.'' Seattle Art Institute, Seattle, WA. May 29, 2014.

\item ``Ask A Scientist Day.'' Highlands Intermediate School, Pearl City, Hawaii. April 24, 2014.

\item ``Exoplanets: How to find them and their inhabitants.'' Boeing Employee Astronomical Society, Seattle, WA. April 10, 2014.

\item ``Transiting Exoplanets: The Meek Telescopes Shall Inherit The Earths.'' Seattle Astronomical Society, Seattle, WA. December 18, 2013.

\item ``Astrobiology In The Age Of Kepler.'' TEDxRainier ``after-day'' at the University of Washington Planetarium, Seattle, WA. November 10, 2013. 

\item ``Exoplanets: How To Find Them And Their Inhabitants.'' Kopernik Observatory and Science Center, Vestal, NY. July 17, 2013. 

\item ``Earthlings: Get Over Yourselves.'' \textit{Astrobiology Speaks!} Series, Redpath Museum, McGill University, Montreal, Canada. June 11, 2013.
\end{itemize}
\vfill \hfill {\small Last updated: \today}
\end{resume}
\end{document}
