% LaTeX resume using res.cls
\documentclass[margin]{res}
%\usepackage{helvetica} % uses helvetica postscript font (download helvetica.sty)
%\usepackage{newcent}   % uses new century schoolbook postscript font 
\setlength{\textwidth}{6.2in} % set width of text portion
\newcommand{\code}[1]{\texttt{#1}}
\usepackage{hyperref}
\hypersetup{
    colorlinks=true,       % false: boxed links; true: colored links
    linkcolor=red,          % color of internal links (change box color with linkbordercolor)
    citecolor=green,        % color of links to bibliography
    filecolor=magenta,      % color of file links
    urlcolor=blue           % color of external links
}

\begin{document}

% Center the name over the entire width of resume:
 \moveleft.5\hoffset\centerline{\textsc{\Large Brett M. Morris}}
 \moveleft.5\hoffset\centerline{Curriculum Vitae}

% Draw a horizontal line the whole width of resume:
\moveleft\hoffset\vbox{\hrule width\resumewidth height 1pt}\smallskip
 \moveleft.5\hoffset\centerline{Email: \href{mailto:bmmorris@uw.edu}{bmmorris@uw.edu}}
 \moveleft.5\hoffset\centerline{Department Website: \url{http://staff.washington.edu/bmmorris}}

%% Select character to use as the bullet marker. 
%% Comment out to get normal dark filled circle bullets
\def\labelitemi{--}	

\begin{resume}

\section{Education} 
University of Washington, Seattle, WA \hfill 2013 -- present\\
Graduate student in Astronomy and Astrobiology (dual-title PhD program)\\
Astrobiology Fellow (2013)

University of Maryland, College Park, MD  \hfill 2009 -- 2012\\
B.S.\ with High Honors in Astronomy \\
B.S.\ in Physics (double degree)\\
 
\section{Research} 

                {\sl NASA Goddard Space Flight Center Research Assistantship} \hfill           Jan 2013 -- Aug 2013\\
                Post-baccalaureate research assistantship with advisor Dr.\ Avi Mandell at the Goddard Center for Astrobiology.
                 \begin{itemize}%  \itemsep -2pt %reduce space between items
                 \item Prepared a Python data reduction pipeline for near-infrared differential spectrophotometric observations with Keck/MOSFIRE and Keck/NIRSPEC of transiting exoplanet atmospheres.
                 \end{itemize}  

                {\sl NASA Goddard Space Flight Center Research Associateship in Astrobiology} \hfill            Jun 2012 -- Aug 2012 \\
                Undergraduate research associateship with mentor Dr.\ Avi Mandell at the Goddard Center for Astrobiology.
                 \begin{itemize}%  \itemsep -2pt %reduce space between items
                 \item Wrote original Python algorithms to compute differential photometry of transiting exoplanet HAT-P-7b to 
                 detect the secondary eclipse using near-infrared observations from the Hale Telescope at the Palomar Observatory. 
                 \item Generated composite light curves with Kepler photometry to measure the orbital parameters and atmospheric 
                 properties of HAT-P-7b, with original analysis code in Python and IDL. 
                 \item Proposed the first evidence for planet-induced stellar gravity darkening in the HAT-P-7 system.
                 \end{itemize}  

                {\sl Undergraduate Research with Professor Drake Deming} \hfill            Aug 2011 -- Dec 2012 \\
                Independent research for credit (ASTR498), three semesters. 
                 \begin{itemize}   
        \item Traveled to Arizona to act as co-investigator on an observing campaign on the 2.1m telescope at Kitt Peak National Observatory for transiting 
        exoplanet observations in the near-infrared. 
                 \item Developed observing techniques and wrote Python algorithms to compute differential photometry 
                 of transiting extrasolar planets using observations obtained at the UMD Observatory, Kitt Peak National Observatory and Palomar Observatory. 
        \item Made the first exoplanet transit light curves produced at the UMD Observatory.
        \item Released and maintain an open source differential photometry code for undergraduate and serious amateur astronomers called ``OSCAAR\footnote{\url{http://oscaar.github.io}}''.
                 \item Submitted transit light curves to the Czech Astronomical Society Exoplanet Transit Database recorded at the University of Maryland Observatory.
                 \end{itemize} 

      {\sl Department of Astronomy Senior Honors Thesis} \hfill Aug 2010 -- Dec 2012 \\
                ``Numerical Modeling of Rotational Fission of Contact Binary Asteroids'' with Professor Derek Richardson, five semesters. 
                 \begin{itemize}   
                 \item Modeled binary asteroid systems in Python and interfaced with an N-body integrator to probe the relative configuration equilibria of continuously torqued 
contact binary asteroid systems. Used resources at the Department of Astronomy's Center for Theory and Computation 
and UMD's High Performance Computing Cluster, computed for more than 4 years of CPU time. 
                \item  Cited with ``High Honors'' by the Department of Astronomy for this work.
                \end{itemize}

                {\sl Undergraduate Research with Prof.\ Christopher Reynolds and Dr.\ Sean O'Neill} \hfill        Spring 2010 \\
                Visualization of magneto-rotational instability in model black hole accretion disks, one semester.
                  \begin{itemize}  
                   \item Produce visualizations with VisIt to emphasize relevant astrophysical processes in model black hole accretion disks in Python. \\
                   \end{itemize} 

\section{Honors\\And\\Awards} 
                 \begin{itemize}   
                  \item Astrobiology Graduate Fellow, University of Washington, 2013-2014.
                  \item ``Audience Choice Award'' at the \textit{Astrobiology Speaks!} public outreach speaking competition. AbGradCon 2013, McGill University, Montreal, Canada. 
                  \item Chambliss Astronomy Achievement Graduate Student Award Honorable Mention. 222$^{\mathtt{nd}}$ AAS, Indianapolis, IN.
		\item Invited to present thesis for Department of Astronomy Honors, awarded High Honors citation (2012).
                  \item UMD Presidential Scholar, 2009-2012.
		\item Feature article in the UMD ``\href{http://www.scholars.umd.edu/news/newsletter/363-february-2012-issue-1}{Scholars Newsletter}'' for research achievements (Feb 2012).
		\item College Park Scholar in the Science, Discovery and the Universe track, 2009-2011 (a two-year honors 
		college program at UMD). \\
                 \end{itemize}

\section{Publications} 
                % {\sl Publications}
                 \begin{itemize}   
                  \item \textbf{Morris, B.M.}, Mandell, A.M., Deming, D. ``\href{http://adsabs.harvard.edu/abs/2013ApJ...764L..22M}{Kepler's Optical Secondary Eclipse of HAT-P-7b and Probable Detection of Planet-Induced Stellar Gravity Darkening}.'' The Astrophysical Journal Letters, Volume 764, Issue 2, article id.\ L22, 5 pp.\ (2013).
                 	%\item \textbf{Morris, B.M.} \& Richardson, D.C. ``Numerical Simulations of Rotational Fission of Contact Binary Asteroids.'' In preparation.
		\item \textbf{Morris, B.M.} ``\href{http://www.scientificterrapin.umd.edu/Fall2011.php}{Observations of Transiting Exoplanets with Differential Photometry}.'' Scientific Terrapin, Vol.\ III, Issue I.\ Fall 2011.\\
                 \end{itemize}

\section{Press\\Releases} 
                 \begin{itemize}
                 \item ``\href{http://www.nasa.gov/content/nasa-funded-program-helps-amateur-astronomers-detect-alien-worlds/}{NASA-funded Program Helps Amateur Astronomers Detect Alien Worlds}''. NASA Goddard Space Flight Center, Greenbelt, Md. September 4, 2013. \\
                 \end{itemize}              

\section{Professional\\ Presentations} 
                % {\sl Publications}
                 \begin{itemize}  %\itemsep -5pt %reduce space between items
                 \item Poster: ``\href{http://nexsci.caltech.edu/conferences/KeplerII/posters/morris.pdf}{Kepler's Optical Secondary Eclipse of HAT-P-7b and Probable Detection of Planet-Induced Stellar Gravity Darkening}.'' Second Kepler Science Conference, NASA Ames Research Center, Mountain View, CA. November 6, 2013. 
                 \item Poster: ``Differential Photometry with OSCAAR: Open Source Differential Photometry Code for Amateur Astronomical Research''. AbGradCon 2013, McGill University, Montreal, Canada. June 12, 2013.
                 \item Talk: ``\href{http://youtu.be/ZMfbkCzzQUE}{Kepler's Optical Secondary Eclipse of HAT-P-7b and Probable Detection of Planet-Induced Stellar Gravity Darkening}''. AbGradCon 2013, McGill University, Montreal, Canada. June 11, 2013.
                 \item Poster: ``\href{http://adsabs.harvard.edu/abs/2013AAS...22221717M}{Differential Photometry with OSCAAR: Open Source Differential Photometry Code for Amateur Astronomical Research}''. 222$^\mathtt{nd}$ American Astronomical Society Meeting. Indianapolis, IN. June 4, 2013.
                 \item Talk: ``\href{https://astrobiology.nasa.gov/seminars/featured-seminar-channels/gsfc-summer-internship/2012/08/09/gsfc-summer-research-associate-presentations/}{Secondary Eclipse and Transiting Timing of Extra-Solar Planet HAT-P-7b}.'' 2012 NASA Goddard Space Flight Center Summer Research Associate Symposium. Goddard Center for Astrobiology, NASA Goddard Space Flight Center, Greenbelt, MD, August 9, 2012.
	        \item Thesis defense:  ``Numerical Simulations of Rotational Fission of Contact Binary Asteroids.'' Department of Astronomy, University of Maryland, College Park, MD. May 4, 2012.
                 \item Abstract: Springmann, S., Dalba, P., Marchis, F., Vachier, F., Berthier, J., Descamps, P., \textbf{Morris B.}, 
		Marciniak, A., Ros, S., Kryszczynska, A. ``\href{http://adsabs.harvard.edu/abs/2012LPICo1667.6352S}{Physical and orbital properties of the (22) Kalliope system from mutual eclipse observations}.'' 
		Asteroids, Comets, Meteors. Japan, May 2012.\\
		\end{itemize}

\section{Observing \\Experience}
            \begin{itemize}   
            \item Kitt Peak National Observatory, 2.1 m (June 2012): Co-investigator for 6 nights observing transiting extrasolar planets in the near-infrared.
            \item University of Maryland Observatory, 152 mm (2010-2013): $>100$ hours collecting photometry of transiting exoplanets and asteroids. \\
            \end{itemize}
                 
\section{Teaching \\Experience}
	     \textbf{University of Washington}
            \begin{itemize}   
            \item Teaching assistant for two sections of ASTR150 The Planets (Fall 2013). 
            \end{itemize}
	     \textbf{University of Maryland}
            \begin{itemize}   
            \item Teaching assistant for ASTR310 Observational Astronomy for majors, experienced teaching night labs at the campus observatory (Fall 2011).
            \item Teaching assistant for ASTR100 Introduction to Astronomy for non-majors (Fall 2012).
            \item Research and service mentor to five undergraduates in the College Park Scholars students, collaborating to build/improve an open source differential photometry code in Python (OSCAAR). \\
            \end{itemize}
            
\section{Public \\ Outreach}             
            \begin{itemize}   
	   \item President of the ``AstroTerps'' (2011-2012), the UMD Astronomical Society.
            \item Volunteer for Visitor Services at the Smithsonian Air and Space Museum in Washington, D.C., May -- Aug 2011
            \item Developed open source differential photometry routine with educational documentation for amateur transiting exoplanet observations (OSCAAR).\\
            \end{itemize}
	   \textbf{Public Talks}
            \begin{itemize}   
            	\item ``Astrobiology In The Age Of Kepler.'' TEDxRainier ``after-day'' at the University of Washington Planetarium, Seattle, WA. November 10, 2013. 
                \item ``Exoplanets: How To Find Them And Their Inhabitants.'' Kopernik Observatory and Science Center, Vestal, NY. July 17, 2013. 
	       \item ``Earthlings: Get Over Yourselves.'' \textit{Astrobiology Speaks!} Series, Redpath Museum, McGill University, Montreal, Canada. June 11, 2013.	   
	       \item ``Studying Transiting Exoplanets with the Biggest and Smallest Telescopes On and Off of Earth.'' University of Maryland Observatory Open House Public Talk. College Park, MD. March 5, 2013.
            \end{itemize}
            
	     \textbf{AstroTerps (UMD Astronomical Society) Talks and Observing Sessions}
	     \begin{itemize}
	     \item Talk and observing session: ``The Leonid Meteor Shower.'' University of Maryland Observatory, College Park, MD. November 17, 2012.
	       \item Talk and observing session: ``Astrophotography and Asteroid (433) Eros.'' University of Maryland Observatory, College Park, MD. February 2, 2012.		
	       \item Talk and observing session: ``The Geminid Meteor Shower.'' University of Maryland Observatory, College Park, MD. December 13, 2011.
	       \item Talk and observing session: ``The Leonid Meteor Shower.'' University of Maryland Observatory, College Park, MD. November 17, 2011.
	       \item Talk: ``Fighting Light Pollution On Campus.'' Department of Astronomy, University of Maryland, College Park, MD. October 11, 2011.
	       \item Talk and observing session: ``Sidewalk Observing On Campus.'' Department of Astronomy, University of Maryland, College Park, MD. September 29, 2011.\\
	       \end{itemize}
\vfill \hfill {\scriptsize Last updated: \today}
\end{resume}
\end{document}
