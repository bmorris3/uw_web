% LaTeX resume using res.cls
\documentclass[margin]{res}
%\usepackage{helvetica} % uses helvetica postscript font (download helvetica.sty)
%\usepackage{newcent}   % uses new century schoolbook postscript font 
\setlength{\textwidth}{6.2in} % set width of text portion
\newcommand{\code}[1]{\texttt{#1}}
\usepackage{hyperref}
\begin{document}

% Center the name over the entire width of resume:
 \moveleft.5\hoffset\centerline{\large\bf Brett M. Morris}

% Draw a horizontal line the whole width of resume:
% \moveleft\hoffset\vbox{\hrule width\resumewidth height 1pt}\smallskip	%% Uncomment this for the separating line at the top
% address begins here
% Again, the address lines must be centered over entire width of resume:
 \moveleft.5\hoffset\centerline{10205 Baltimore Ave, APT 7306, College Park, MD 20740}
 \moveleft.5\hoffset\centerline{Email: \href{mailto:brett.m.morris@nasa.gov}{brett.m.morris@nasa.gov}, Mobile: (631) 860-5116}
 \moveleft.5\hoffset\centerline{Department Website: \url{http://www.astro.umd.edu/~bmorris3/}}

\begin{resume}

\section{Education} 
	   Graduate student of astronomy at the University of Washington, Seattle, WA. Enrolled in dual-title astrobiology program.\\\\
	     Bachelor of Science with High Honors in Astronomy from the Dept. of Astronomy, \hfill 2009-2012\\
	     Bachelor of Science in Physics (Double Degree) from University of Maryland, College Park, MD.\\
                      % \sl will be bold italic in New Century Schoolbook (or
	              % any postscript font) and just slanted in
		      %	Computer Modern (default) font
 
\section{Computer \\ Skills} {\sl Languages:} Python, IDL, Matlab, HTML.  {\sl Operating Systems:} Unix, Mac OS,  MS. \\
 
\section{Research} 

                {\sl NASA Goddard Space Flight Center Research Assistantship} \hfill           January 2013 -- present \\
                Post-baccalaureate research assistantship with mentor Dr. Avi Mandell at the Goddard Center for Astrobiology.
                 \begin{itemize}%  \itemsep -2pt %reduce space between items
                 \item Prepared a Python data reduction pipeline for near-infrared differential spectrophotometric observations with Keck/MOSFIRE of transiting exoplanet atmospheres.
                 \end{itemize}  

                {\sl NASA Goddard Space Flight Center Research Associateship in Astrobiology} \hfill            June -- August 2012 \\
                Undergraduate research associateship with mentor Dr. Avi Mandell at the Goddard Center for Astrobiology.
                 \begin{itemize}%  \itemsep -2pt %reduce space between items
                 \item Wrote original Python algorithms to compute differential photometry of transiting exoplanet HAT-P-7b to 
                 detect the secondary eclipse using near-infrared observations from the Hale Telescope at the Palomar Observatory. 
                 \item Generated composite light curves with Kepler photometry to measure the orbital parameters and atmospheric 
                 properties of HAT-P-7b, with original analysis code in Python and IDL. 
                 \item Found the first evidence for planet-induced stellar gravity darkening in the HAT-P-7 system.
                 \end{itemize}  

                {\sl Undergraduate Research with Professor Drake Deming} \hfill            Aug 2011 -- Dec 2012 \\
                Independent research for credit (ASTR498), three semesters. 
                 \begin{itemize}   
	        \item Traveled to Arizona to act as co-investigator on an observing campaign on the 2.1m telescope at Kitt Peak National Observatory for transiting 
	        exoplanet observations in the near-infrared. 
                 \item Developed observing techniques and wrote Python algorithms to compute differential photometry 
                 of transiting extrasolar planets using observations obtained at the UMD Observatory, Kitt Peak National Observatory and Palomar Observatory. 
	        \item Made the first exoplanet transit light curves produced at the UMD Observatory.
	        \item Released and maintain an open source differential photometry code for undergraduate and serious amateur astronomers called ``OSCAAR\footnote{\url{http://oscaar.github.io}}''.
                 \item Submitted transit light curves to the Czech Astronomical Society Exoplanet Transit Database recorded at the University of Maryland Observatory.
                 \end{itemize} 

	      {\sl Department of Astronomy Senior Honors Thesis} \hfill Aug 2010 -- Dec 2012 \\
                ``Numerical Modeling of Rotational Fission of Contact Binary Asteroids'' with Professor Derek Richardson, five semesters. 
                 \begin{itemize}   
                 \item Modeled binary asteroid systems in Python and interfaced with an N-body integrator to probe the relative configuration equilibria of continuously torqued 
		contact binary asteroid systems. Used resources at the Department of Astronomy's Center for Theory and Computation 
		and UMD's High Performance Computing Cluster, computed for more than 4 years of CPU time. 
                \item  Cited with ``High Honors'' by the Department of Astronomy for this work.
                \end{itemize}

                {\sl Undergraduate Research with Christopher Reynolds and Sean O'Neill} \hfill        Spring 2010 \\
                Visualization of magneto-rotational instability in model black hole accretion disks, one semester.
                  \begin{itemize}  
                   \item Produce visualizations with VisIt to emphasize relevant astrophysical processes in model black hole accretion disks in Python. 
                   \item Acknowledged in the resultant publication in ApJ, ``Low-Frequency Oscillations in Global Simulations of Black Hole Accretion'' by  O'Neill et al.\ (2012). \\
                   \end{itemize} 

\section{Honors} 
                 %{\sl Honors}
                 \begin{itemize}   
                  \item Presidential Scholar, 2009-2012.
		\item College Park Scholar in the Science, Discovery and the Universe track, 2009-2011 (a two-year honors 
		college program at UMD). 
		\item Invited to present thesis for Department of Astronomy Honors, received High Honors citation.
		\item Feature article in ``Scholars Newsletter'' for research achievements (Feb 2012).\\
                 \end{itemize}
                 
\section{Publications} 
                % {\sl Publications}
                 \begin{itemize}   
                  \item \textbf{Morris, B.M.}, Mandell, A.M., Deming, D. ``Kepler's Optical Secondary Eclipse of HAT-P-7b and Probable Detection of Planet-Induced Stellar Gravity Darkening.'' The Astrophysical Journal Letters, Volume 764, Issue 2, article id. L22, 5 pp. (2013).
                 	%\item \textbf{Morris, B.M.} \& Richardson, D.C. ``Numerical Simulations of Rotational Fission of Contact Binary Asteroids.'' In preparation.
		\item \textbf{Morris, B.M.} ``Observations of Transiting Exoplanets with Differential Photometry.'' Scientific Terrapin, Vol. III, Issue I. Fall 2011.\\
                 \end{itemize}

\section{Professional\\ Presentations} 
                % {\sl Publications}
                 \begin{itemize}  %\itemsep -5pt %reduce space between items
                 \item Poster: ``Differential Photometry with OSCAAR: Open Source Differential Photometry Code for Amateur Astronomical Research''. AbGradCon 2013, McGill University, Montreal, Canada. June 12, 2013.
                 \item Talk: ``Kepler�s Optical Secondary Eclipse of HAT-P-7b and Probable Detection of Planet-Induced Stellar Gravity Darkening''. AbGradCon 2013, McGill University, Montreal, Canada. June 11, 2013.
                 \item Poster: ``Differential Photometry with OSCAAR: Open Source Differential Photometry Code for Amateur Astronomical Research''. 222$^\mathtt{nd}$ American Astronomical Society Meeting. Indianapolis, IN. June 4, 2013.
                 \item Symposium: ``Secondary Eclipse and Transiting Timing of Extra-Solar Plant HAT-P-7b''. 2012 NASA Goddard Space Flight Center Summer Research Associate Symposium. Goddard Center for Astrobiology, NASA Goddard Space Flight Center, Greenbelt, MD, August 9, 2012.
	        \item Thesis defense:  ``Numerical Simulations of Rotational Fission of Contact Binary Asteroids.'' Department of Astronomy, University of Maryland, College Park, MD. May 4, 2012.
                 \item Abstract: Springmann, S., Dalba, P., Marchis, F., Vachier, F., Berthier, J., Descamps, P., \textbf{Morris B.}, 
		Marciniak, A., Ros, S., Kryszczynska, A. ``Physical and orbital properties of the (22) Kalliope system from mutual eclipse observations.'' 
		Asteroids, Comets, Meteors. Japan, May 2012.\\
		\end{itemize}

\section{Observing \\Experience}
            \begin{itemize}   
            \item Kitt Peak National Observatory, 2.1 m (June 2012): Co-investigator for 6 nights observing transiting extrasolar planets in the near-infrared.
            \item University of Maryland Observatory, 6 in (2010-present): $>100$ hours collecting photometry of transiting exoplanets and asteroids. \\
            \end{itemize}
                 
\section{Teaching \\Experience}
            \begin{itemize}   
            \item Teaching assistant for ASTR310 Observational Astronomy for majors, experienced teaching night labs at the campus observatory (Fall 2011).
            \item Teaching assistant for ASTR100 Introduction to Astronomy for non-majors (Fall 2012).
            \item Research and service mentor to six undergraduates in the College Park Scholars students, collaborating to build/improve an open source differential photometry code in Python.\\
            \end{itemize}
            
\section{Public \\ Outreach}             
            \begin{itemize}   
	   \item President of the ``AstroTerps'' (2011-2012), the UMD Astronomical Society.
            \item Volunteer for Visitor Services at the Smithsonian Air and Space Museum in Washington, D.C.\\
            \end{itemize}
	   \textbf{Public Talks}
	   
            \begin{itemize}   
	       \item Talk: ``Earthlings: Get Over Yourselves.'' Astrobiology Speaks! Series, Redpath Museum, McGill University, Montreal, Canada. June 11, 2013.	   
	       \item Talk: ``Studying Transiting Exoplanets with the Biggest and Smallest Telescopes On and Off of Earth.'' University of Maryland Observatory Open House Public Talk. College Park, MD. March 5, 2013.
            \end{itemize}
            
	     \textbf{AstroTerps (UMD Astronomical Society) Talks and Observing Sessions}
	     \begin{itemize}
	     \item Talk (45 min) and observing session: ``The Leonid Meteor Shower.'' University of Maryland Observatory, College Park, MD. November 17, 2012.
	       \item Talk (45 min) and observing session: ``Astrophotography and Asteroid (433) Eros.'' University of Maryland Observatory, College Park, MD. February 2, 2012.		
	       \item Talk (45 min) and observing session: ``The Geminid Meteor Shower.'' University of Maryland Observatory, College Park, MD. December 13, 2011.
	       \item Talk (45 min) and observing session: ``The Leonid Meteor Shower.'' University of Maryland Observatory, College Park, MD. November 17, 2011.
	       \item Talk (45 min): ``Fighting Light Pollution On Campus.'' Department of Astronomy, University of Maryland, College Park, MD. October 11, 2011.
	       \item Talk (45 min) and observing session: ``Sidewalk Observing On Campus.'' Department of Astronomy, University of Maryland, College Park, MD. September 29, 2011.\\
	       \end{itemize}
            
\section{References}         
	Dr. Avi Mandell			 \hfill		avi.mandell@nasa.gov\\
	Professor Derek Richardson 	 \hfill		dcr@astro.umd.edu \\
	Professor L. Drake Deming           \hfill		lddeming@gmail.com 

\end{resume}
\end{document}