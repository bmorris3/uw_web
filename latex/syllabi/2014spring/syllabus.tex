\documentclass[12pt]{article}
\textwidth=7in
\textheight=9.5in
\topmargin=-1in
\headheight=0in
\headsep=.5in
\hoffset  -.85in

\usepackage{hyperref}
\hypersetup{
    colorlinks=true,       % false: boxed links; true: colored links
    linkcolor=red,          % color of internal links (change box color with linkbordercolor)
    citecolor=green,        % color of links to bibliography
    filecolor=magenta,      % color of file links
    urlcolor=blue           % color of external links
}

\newcommand{\mytitle}[1]{\vspace{5mm} \noindent{\bf #1:} ~~}

\pagestyle{empty}
\begin{document}

\begin{center}
{\bf Astronomy 150 -- The Planets \\  Spring 2014 Syllabus}
\end{center}

\setlength{\unitlength}{1in}

\begin{picture}(6,.1) 
\put(0,0) {\line(1,0){6.25}}         
\end{picture}\\

\noindent
{\bf Teaching Assistant:} Brett Morris\\
{\bf Email:} bmmorris@uw.edu\\
{\bf My Office:} PAB B337\\
{\bf Office Hours:} Tuesday and Thursday from 11:30-12:30pm (after lecture), or by appointment\\
{\bf Course Website:} astro150.org\\
{\bf Section Website:} goo.gl/QeRBXm\\
{\bf Section Room:} PAA A216 \\
%\begin{center}

\begin{picture}(6,.1) 
\put(0,0) {\line(1,0){6.25}}         
\end{picture}\\

\noindent
Welcome to AST 150! Here are a few reiterations of Dr.~Smith's syllabus pertaining to section. \\

\mytitle{Lecture Materials} Don't miss Dr.~Smith's lectures and take careful notes in them. Said many students: ``But Brett, there are no words on Dr. Smith's slides, so I can't possibly learn from them if I didn't go to lecture.'' Said Brett for the $N$-hundredth time: ``Told you so.''

\mytitle{Section Materials} Bring your lab course packet with you to every section. 

\mytitle{Section Attendance} Don't miss sections. The labs that we will do in sections usually can not be completed successfully at home. They count for 30\% of your grade and more than 30\% of the actual learning that you will do in the course. If you miss sections due to illness, contact me on the day that you missed sections to arrange a make up in a timely fashion. 

{\bf Late labs will not be accepted}. From Dr. Smith's syllabus: ``\textit{No late work will be accepted in section. Missing a lab will result in zero credit for that lab.}''

\mytitle{Office Hours} My office (B337) is in the Physics and Astronomy Building (PAB) on the third floor. Look for the ``B. Morris'' tag next to my door in the long hallway of graduate students, or find me on the \href{http://staff.washington.edu/bmmorris/docs/astromap.pdf}{Astronomy Department Finding Chart} ahead of time so you don't get lost. You can find me in my office on Tuesday and Thursday from 11:30-12:30pm, or if those times don't work for you, send me an email and we'll find one that works. 

Office hours are your chance to get help with assignments, ask questions before exams, and learn more about astronomy. Don't wait until the day before the test to ask a question -- come to office hours!

\end{document}
\noindent
{\bf Important Dates}
\begin{center}
    \begin{tabular}{c|c|c}
    Exam & Review Date & Exam Date \\ \hline
    Midterm 1 & Sept 28 & Oct 1 \\
    Midterm 2 & Oct 16 & Oct 22 \\
    Midterm 3 & Nov 15 & Nov 16\\
    Final & Dec 15 or 16 & Dec 17 (8-10am)\\
    \end{tabular}\\
\end{center}