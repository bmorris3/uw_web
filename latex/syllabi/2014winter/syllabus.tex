\documentclass[12pt]{article}
\textwidth=7in
\textheight=9.5in
\topmargin=0in
\headheight=0in
\headsep=.5in
\hoffset  -.85in

\pagestyle{empty}
\begin{document}

\begin{center}
{\bf Astronomy 101 \\ Winter 2014 Syllabus}
\end{center}

\setlength{\unitlength}{1in}

\begin{picture}(6,.1) 
\put(0,0) {\line(1,0){6.25}}         
\end{picture}\\

\noindent
{\bf Teaching Assistant:} Brett Morris\\
{\bf Email:} bmmorris@uw.edu\\
{\bf My Office:} PAB B337\\
{\bf Office Hours:} Mondays from 11:30-12:30 and 3-4 pm, or by appointment\\
{\bf Section Website:} goo.gl/QeRBXm\\
{\bf Section Room:} PAA A216 \\
%\begin{center}

\begin{picture}(6,.1) 
\put(0,0) {\line(1,0){6.25}}         
\end{picture}\\

\noindent
Welcome to AST 101! Here are some guidelines for how I will run our sections and a few reiterations 
of Dr. Balick's syllabus pertaining to section. \\
%\end{center}

\noindent
{\bf Office Hours: }  
My office (B337) is in the Physics and Astronomy Building (PAB) on the third floor,
look for the ``B. Morris'' tag next to my door. You can find me in my office on
Mondays from 11:30-12:30 and 3-4 pm, or if those times don't work for you, 
send me an email and we'll find one that works. 

Office hours are your chance to get help with assignments, ask questions before
exams, and learn more about astronomy. Don't wait until the day before an exam 
to ask a question!\\

\noindent
{\bf Section Work: }
Your work in sections counts for \textbf{40\% of your grade} in this course, nearly double the midterm
or the final, so it is best that you take it seriously. I will not directly take attendance at each section,
however graded assignments will be given during each section. If you do not attend the section, you will miss a 
graded assignment. \\

\noindent
{\bf Late Work: }
Unexcused late work is very strongly discouraged, so the penalties incurred for late assignments are harsh.
Assignments are due in the first few minutes of each section. I will accept work on the day that it was due 
in my mailbox on the third floor of the Physics and Astronomy Building until 5pm. After the 5pm cutoff,
no further late work will be accepted. 

Regarding late work in Dr. Balick's syllabus: ``\textit{Late work is accepted under two conditions: You get our permission and we agree on a final due date.  Several of the exercises require special equipment so there is no way to make these up. No late work will be accepted after March 11.}'' \\


\noindent
{\bf Group work: }
It is often extremely helpful to do homework assignments in courses like this one in groups. 
I encourage working with your classmates on homework, so long as the assignments you turn in
are unique and independently written up. The best practice for group work without the threat 
of plagiarism is to work out the problems together once, and write up the copy that you will
turn in separately. No two assignments that I grade should be identical.\\


\end{document}
\noindent
{\bf Important Dates}
\begin{center}
    \begin{tabular}{c|c|c}
    Exam & Review Date & Exam Date \\ \hline
    Midterm 1 & Sept 28 & Oct 1 \\
    Midterm 2 & Oct 16 & Oct 22 \\
    Midterm 3 & Nov 15 & Nov 16\\
    Final & Dec 15 or 16 & Dec 17 (8-10am)\\
    \end{tabular}\\
\end{center}
